En el contexto de ingeniería mecatrónica e ingeniería de control, el estudio de drones resulta de particular interés debido a su versatilidad y capacidad de implementación de algortimos de control en entornos dinámicos y cambiantes. Hace algunos años, la Universidad del Valle de Guatemala adquirió un conjunto de drones Crazyflie, con lo que inició una línea de investigación para adaptar su uso y control en el entorno académico. Desde entonces, se han llevado a cabo distintos proyectos que han buscado mejorar el uso de los mismos empleando distintas técnicas. A pesar de los avances alcanzados en el uso de drones Crazyflie, las técnicas actuales resultan complejas y poco prácticas de utilizar en un entorno académico debido a su alta curva de aprendizaje. 

En este protocolo se propone una alternativa para utilizar los drones Crazyflie de manera individual, sencilla y práctica. Dado que ya se han explorado algunas vías para su uso, se optará por una alternativa de control que no ha sido empleada anteriormente y que consiste en la integración de una placa de expansión para posicionamiento de los drones mediante odometría visual. Con ello en mente, se busca maximizar el uso de los drones, aprovechando su potencial y facilitando su implementación en cursos de ingeniería mecatrónica y electrónica, tales como sistemas de control 1 y 2. 