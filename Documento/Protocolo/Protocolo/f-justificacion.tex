Los avances en investigación de ingeniería de control han sido significativamente influenciados por el uso de drones, que se han convertido en herramientas indispensables debido a su alta capacidad para implementar algoritmos de control. Su versatilidad y facilidad de adaptación hacen de los drones un recurso valioso para la enseñanza de cursos relacionados con ingeniería de control y robótica. 

En la Universidad del Valle de Guatemala, se han desarrollado distintos proyectos de investigación que han aprovechado las capacidades de los drones Crazyflie. Estos proyectos han explorado distintas aplicaciones, desde su uso en plataformas de pruebas hasta la formación de enjambres de drones en entornos controlados. 

Recientemente, se ha desarrollado un conjunto de herramientas que, si bien demostraron ser efectivas, resultan complicadas de utilizar debido a su alta curva de aprendizaje y escasa documentación. En particular, la herramienta Crazyswarm 2 en ROS2 ha demostrado ser efectiva pero difícil de manejar debido a los requerimientos de conocimientos avanzados en Linux y ROS2, así como la falta de documentación disponible. Por ello, surge la necesidad de simplificar las herramientas para facilitar el uso práctico y seguro de los drones.

El propósito de este trabajo de graduación se centra en la oportunidad de ampliar el uso de los drones Crazyflie en la Universidad del Valle de Guatemala, permitiendo su uso independiente y seguro mediante la integración de la placa de expansión para posicionamiento Flow Deck. De esta manera, los drones podrán ser utilizados de forma sencilla en prácticas de laboratorio en cursos de ingeniería electrónica y mecatrónica, como sistemas de control 1 y 2.  
