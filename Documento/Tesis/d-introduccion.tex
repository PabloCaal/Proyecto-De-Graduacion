En los últimos años, los drones han incrementado su relevancia en diversas áreas, desde la investigación científica hasta aplicaciones industriales. Esta relevancia se debe a su notable capacidad de adaptación a distintos entornos y a las soluciones innovadoras que pueden ofrecer en un amplio rango de aplicaciones.

La Universidad del Valle de Guatemala dispone de un conjunto de micro-drones Crazyflie, adquiridos desde hace algunos años con fines educativos. Además, dispone de un laboratorio equipado con un ecosistema adaptado para estudios de ingeniería mecatrónica y electrónica, donde se realizan prácticas de laboratorio que involucran agentes autónomos como robots móviles, brazos robóticos y drones. 

En este proyecto se explora una nueva alternativa de control para los drones Crazyflie 2.1 de la Universidad del Valle de Guatemala, implementando un sistema de posicionamiento basado en odometría visual. Además, se desarrollan herramientas de \textit{software} para facilitar la maniobrabilidad de los drones con el sistema de posicionamiento. Y, adicionalmente, se elabora un manual de usuario y guías de laboratorios orientadas a los cursos de Robótica y Sistemas de Control.