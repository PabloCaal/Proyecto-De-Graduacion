Este proyecto tiene como alcance la integración y validación de la placa de expansión Flow Deck en los drones Crazyflie 2.1 de la Universidad del Valle de Guatemala, utilizando las herramientas y librerías otorgadas por el fabricante. Además, se busca desarrollar herramientas de \textit{software} simplificadas que permitan el control básico de los drones y faciliten su uso dentro de un entorno educativo de laboratorio.

Se propone también la creación de dos guías de laboratorio para cursos impartidos en la Universidad del Valle de Guatemala. La primera para el curso de Sistemas de Control 1, enfocada en la modificación del controlador PID de posición para el control de altura del dron. La segunda para el curso de Robótica 1, en la que se desarrollarán experimentos de seguimiento de trayectorias con obstáculos. Finalmente, se elaborará un manual de usuario para el Crazyflie 2.1 con la placa Flow Deck integrada, para que los usuarios puedan familizarizarse con su configuración y manejo. 