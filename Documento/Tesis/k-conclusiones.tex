\begin{itemize}
	\item Se realizó la integración de la placa de expansión Flow Deck en el dron Crazyflie 2.1, validando su funcionamiento mediante un conjunto de pruebas de vuelo y posicionamiento simples. En el proceso, se identificó una superifice con las condiciones para el correcto funcionamiento de los sensores presentes en la placa Flow Deck.
	
	\item Se desarrollaron herramientas de \textit{software} para controlar y monitorear de forma simple al dron Crazyflie con la placa Flow Deck incorporada. Estas funciones permiten ejecutar desde experimentos simples de despegue, hasta seguimientos de trayectorias.
	
	\item Se desarrolló un manual de usuario para el dron Crazyflie 2.1 con la placa de expansión Flow Deck integrada, proporcionando a los usuarios una guía detallada sobre el ensamble, configuración y manejo del dron. 	
	
	\item Se desarrolló una guía de laboratorio diseñada para el curso de Sistemas de Control 1, donde se utilizó al Crazyflie 2.1 con la placa Flow Deck para evaluar el comportamiento de vuelo al modificar las constantes del controlador PID de altura del dron.
	
	\item Se desarrolló una guía de laboratorio diseñada para el curso de Robótica 1, donde se utilizó al Crazyflie 2.1 con la placa Flow Deck para la generación y seguimiento de trayectorias a través de una pista de obstáculos sobre el ecosistema de investigación Robotat.
\end{itemize}
