\section{Repositorio en GitHub}
Puede acceder al repositorio en Github por medio del siguiente enlace: \url{https://github.com/PabloCaal/Crazyflie-Matlab-Python}

\section{Manual de usuario para Crazyflie 2.1 con la placa de expansión Flow Deck}
Puede acceder al documento del manual de usuario por medio del siguiente enlace: \url{https://drive.google.com/file/d/157fDCxXf_BLKTRiDAqIZCCapgSy8fdp2/view?usp=sharing}

\section{Laboratorio 1: Control PID de altura}
Puede acceder al documento del laboratorio 1 por medio del siguiente enlace: \url{https://drive.google.com/file/d/1cBeLzd2d4w1BNKShUpOJaZk3qm76pnpr/view?usp=sharing}

\section{Laboratorio 2: Seguimiento de trayectorias}
Puede acceder al documento del laboratorio 2 por medio del siguiente enlace: \url{https://drive.google.com/file/d/110EhRG25IjRaHUYy3AXFovrk0BIy7Ew8/view?usp=sharing}

\newpage
\section{Funciones de control desarrolladas en Python}
\subsection{Función connect}
\begin{lstlisting}[caption=Función en Python para establecer la conexión con Crazyflie., label=code:funcion_connect]
	def connect(uri):
		try:        
			scf = SyncCrazyflie(uri, cf=Crazyflie(rw_cache='./cache'))
			scf.open_link()
			print(f"Connection to Crazyflie established successfully.")
			sys.stdout.flush()
			return scf
			
		except Exception as e:
			if 'Cannot find a Crazyradio Dongle' in str(e):
				print(f"Error: Crazyradio Dongle not found. Ensure the dongle is connected properly.")
			elif 'Connection refused' in str(e):
				print(f"Error: Connection to Crazyflie was refused. Check if the Crazyflie is powered on and in range.")
			else:
				print(f"General error occurred while trying to connect to Crazyflie. Error details: {str(e)}")
\end{lstlisting}

\subsection{Función disconnect}
\begin{lstlisting}[caption=Función en Python para cerrar la conexión con Crazyflie., label=code:funcion_disconnect]
	def disconnect(scf):
		try:
			if scf:
				scf.close_link()
				print(f"Successfully disconnected from Crazyflie.")
			else:
				print(f"Error: Invalid SyncCrazyflie object. No connection to close.")
				
		except Exception as e:
			print(f"Error: An issue occurred while disconnecting from Crazyflie. Error details: {str(e)}")
\end{lstlisting}

\newpage
\subsection{Función get\_pose}
\begin{lstlisting}[caption=Función en Python para obtener la pose actual del Crazyflie., label=code:funcion_get_pose]
	def get_pose(scf):
		try:
			# Set up the log configuration to get position and orientation data
			pose_log_config = LogConfig(name='Pose', period_in_ms=100)
			pose_log_config.add_variable('stateEstimate.x', 'float')
			pose_log_config.add_variable('stateEstimate.y', 'float')
			pose_log_config.add_variable('stateEstimate.z', 'float')
			pose_log_config.add_variable('stateEstimate.roll', 'float')
			pose_log_config.add_variable('stateEstimate.pitch', 'float')
			pose_log_config.add_variable('stateEstimate.yaw', 'float')
			pose = {'x': 0.0, 'y': 0.0, 'z': 0.0, 'roll': 0.0, 'pitch': 0.0, 'yaw': 0.0}
			new_data = Event()
			
			def pose_callback(timestamp, data, logconf):
				pose['x'] = data['stateEstimate.x']
				pose['y'] = data['stateEstimate.y']
				pose['z'] = data['stateEstimate.z']
				pose['roll'] = data['stateEstimate.roll']
				pose['pitch'] = data['stateEstimate.pitch']
				pose['yaw'] = data['stateEstimate.yaw']
				new_data.set()
			
			pose_log_config.data_received_cb.add_callback(pose_callback)
			
			try:
				existing_configs = scf.cf.log.log_blocks
				for config in existing_configs:
					if config.name == 'Pose':
						config.stop()
						config.delete()
						
			except AttributeError:
				pass  
				
			scf.cf.log.add_config(pose_log_config)
			pose_log_config.start()
			new_data.wait()
			pose_log_config.stop()
			print(f"Pose retrieved successfully")
			print(f"x: {pose['x']:.2f}, y: {pose['y']:.2f}, z: {pose['z']:.2f}, roll: {pose['roll']:.2f}, pitch: {pose['pitch']:.2f}, yaw: {pose['yaw']:.2f}")
			return [pose['x'], pose['y'], pose['z'], pose['roll'], pose['pitch'], pose['yaw']]
		
		except Exception as e:
			print(f"ERROR: An error occurred while retrieving the pose: {str(e)}")
\end{lstlisting}

\newpage
\subsection{Función get\_pid\_values}
\begin{lstlisting}[caption=Función en Python para obtener los valores de todos los PID de posición del Crazyflie., label=code:funcion_get_pid_values]
	def get_pid_values(scf):
		try:
			pid_values = {
				'X': [
				float(scf.cf.param.get_value('posCtlPid.xKp')),
				float(scf.cf.param.get_value('posCtlPid.xKi')),
				float(scf.cf.param.get_value('posCtlPid.xKd'))],
				'Y': [
				float(scf.cf.param.get_value('posCtlPid.yKp')),
				float(scf.cf.param.get_value('posCtlPid.yKi')),
				float(scf.cf.param.get_value('posCtlPid.yKd'))],
				'Z': [
				float(scf.cf.param.get_value('posCtlPid.zKp')),
				float(scf.cf.param.get_value('posCtlPid.zKi')),
				float(scf.cf.param.get_value('posCtlPid.zKd'))]
			}
			print("PID values for X axis: P={:.2f}, I={:.2f}, D={:.2f}".format(pid_values['X'][0], pid_values['X'][1], pid_values['X'][2]))
			print("PID values for Y axis: P={:.2f}, I={:.2f}, D={:.2f}".format(pid_values['Y'][0], 	pid_values['Y'][1], pid_values['Y'][2]))
			print("PID values for Z axis: P={:.2f}, I={:.2f}, D={:.2f}".format(pid_values['Z'][0], 	pid_values['Z'][1], pid_values['Z'][2]))
			return pid_values
		
		except Exception as e:
			print(f"ERROR: An error occurred during the pid values lecture: {str(e)}")
\end{lstlisting}

\subsection{Función get\_pid\_x}
\begin{lstlisting}[caption=Función en Python para obtener un PID de posición específico del Crazyflie., label=code:funcion_get_pid_x]
	def get_pid_x(scf):
		try:
			pid_x = {
				'P': float(scf.cf.param.get_value('posCtlPid.xKp')),
				'I': float(scf.cf.param.get_value('posCtlPid.xKi')),
				'D': float(scf.cf.param.get_value('posCtlPid.xKd'))
			}
			print("PID values for X axis: P = {:.2f}, I = {:.2f}, D = {:.2f}".format(pid_x['P'], pid_x['I'], 	pid_x['D']))
			return pid_x
		
		except Exception as e:
			print(f"ERROR: An error occurred while retrieving the PID values for X axis: {str(e)}")
\end{lstlisting}

\newpage
\subsection{Función detect\_flow\_deck}
\begin{lstlisting}[caption=Función en Python para detectar la placa Flow Deck en Crazyflie., label=code:funcion_detect_flow_deck]
	def detect_flow_deck(scf):
		try:
			flow_deck_detected = scf.cf.param.get_value('deck.bcFlow2')
			if flow_deck_detected == '1':
				print(f"Flow Deck detected successfully.")
				return 1
			else:
				print(f"Flow Deck not detected. Please verify that it is installed properly.")
				return 0
		
		except Exception as e:
			print(f"Error: An issue occurred while detecting the Flow Deck. Error details: {str(e)}")
\end{lstlisting}

\subsection{Función set\_position}
\begin{lstlisting}[caption=Función en Python para actualizar la posición del estimador del Crazyflie., label=code:funcion_set_position]
	def set_position(scf, x, y, z):
		try:
			if not all(isinstance(coord, (int, float)) for coord in [x, y, z]):
				print(f"ERROR: Input values invalids.")
			scf.cf.extpos.send_extpos(x, y, z)
			time.sleep(0.01)
			print(f"Absolute position successfully set.")
		
		except Exception as e:
			print(f"ERROR: An error occurred during the position update: {str(e)}")
\end{lstlisting}

\subsection{Función set\_pid\_x}
\begin{lstlisting}[caption=Función en Python para configurar a un PID de posición específico del Crazyflie., label=code:funcion_set_pid_x]
	def set_pid_x(scf, P, I, D):
		try:      
			scf.cf.param.set_value('posCtlPid.xKp', P)
			scf.cf.param.set_value('posCtlPid.xKi', I)
			scf.cf.param.set_value('posCtlPid.xKd', D)
			print(f"Successful PID modification.")
		
		except Exception as e:
			print(f"ERROR: An error occurred during the PID modification: {str(e)}")
\end{lstlisting}

\newpage
\subsection{Función set\_pid\_values}
\begin{lstlisting}[caption=Función en Python para configurar todos los PID de posición del Crazyflie., label=code:funcion_set_pid_values]
	def set_pid_values(scf, p_gains, i_gains, d_gains):
		try:       
			scf.cf.param.set_value('posCtlPid.xKp', p_gains['X'])
			scf.cf.param.set_value('posCtlPid.xKi', i_gains['X'])
			scf.cf.param.set_value('posCtlPid.xKd', d_gains['X'])
			scf.cf.param.set_value('posCtlPid.yKp', p_gains['Y'])
			scf.cf.param.set_value('posCtlPid.yKi', i_gains['Y'])
			scf.cf.param.set_value('posCtlPid.yKd', d_gains['Y'])
			scf.cf.param.set_value('posCtlPid.zKp', p_gains['Z'])
			scf.cf.param.set_value('posCtlPid.zKi', i_gains['Z'])
			scf.cf.param.set_value('posCtlPid.zKd', d_gains['Z'])
			print(f"Successful PID modification.")
		
		except Exception as e:
			print(f"ERROR: An error occurred during the PID modification: {str(e)}")
\end{lstlisting}

\subsection{Función takeoff}
\begin{lstlisting}[caption=Función en Python para despegar al Crazyflie., label=code:funcion_takeoff]
	def takeoff(scf, height = 0.3, duration = 1.0):
		try:
			position = get_pose(scf)
			current_z = position[2]  
			if current_z > 0.1:
				print(f"The Crazyflie was already in the air.")
				return 0
			commander = HighLevelCommander(scf.cf)
			commander.takeoff(absolute_height_m=height, duration_s=duration)
			time.sleep(duration)
			print(f"Takeoff completed successfully")
		
		except Exception as e:
			print(f"ERROR: An error occurred during takeoff: {str(e)}")
\end{lstlisting}

\newpage
\subsection{Función land}
\begin{lstlisting}[caption=Función en Python para aterrizar al Crazyflie., label=code:funcion_land]
	def land(scf, height = 0.0, duration = 2.0):
		try:
			position = get_pose(scf)
			current_z = position[2]  
			if current_z <= 0.1:
				print(f"The Crazyflie was already on the ground.")
				return 0
			commander = HighLevelCommander(scf.cf)
			commander.land(absolute_height_m=height, duration_s=duration)
			time.sleep(duration)
			commander.stop()
			print(f"Landing completed successfully.")
		
		except Exception as e:
			print(f"ERROR: An error occurred during landing: {str(e)}")
\end{lstlisting}

\subsection{Función move\_to\_position}\begin{lstlisting}[caption=Función en Python para enviar a una posición específica al Crazyflie., label=code:funcion_move_to_position]
	def move_to_position(scf, x, y, z, velocity = 1.0):
		try:
			commander = scf.cf.high_level_commander
			current_position = get_pose(scf)
			current_x, current_y, current_z = current_position[0], current_position[1], current_position[2]
			distance = ((x - current_x)**2 + (y - current_y)**2 + (z - current_z)**2)**0.5
			duration = distance / velocity
			commander.go_to(x, y, z, yaw=0.0, duration_s=duration)
			time.sleep(duration)
			print(f"Position command completed successfully")
		
		except Exception as e:
			print(f"ERROR: An error occurred during moving to position: {str(e)}")
\end{lstlisting}